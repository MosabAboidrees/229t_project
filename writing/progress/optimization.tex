\subsubsection*{Optimization}

Thus far we have experimented with several optimizers, all of which are
variations of the basic stochastic gradient descent (SGD) optimization
procedure. Let $\ell(\theta)$ be the loss function of the network with
parameters $\theta$. We use accelerated SGD methods that update the parameters
at time $t$ according to
\begin{align*}
&v_{t+1} = \mu v_t - \eta \nabla \ell (\theta_t + \gamma v_t) \\
&\theta_{t+1} = \theta_t + v_{t+1}
\end{align*}
where $\eta$ is the learning rate, and the parameter $\mu$, known as the momentum
parameter, dictates how much gradient history we take into account at every
update. If we set $\mu = 0$ we recover plain SGD, and setting $\mu = 1$ uses
the full gradient history at every update.  The $\gamma$ parameter is either
active and set as $\gamma = \mu$ or inactive and set as $\gamma = 0$.  When
$\gamma$ is inactive, we have typical SGD with ``momentum'', also known as
classical momentum (CM). On the other hand, when $\gamma$ is active the
procedure is known as Nesterov's accelerated graient (NAG).

We also study the adaptive gradient algorithm (AdaGrad) with a diagonal
preconditioning matrix $G$. The update can be written as
\begin{align*}
&G_{t+1} = G_{t} + \diag(\nabla \ell (\theta_t))^2 \\
&\theta_{t+1} = \theta_t - G_{t+1}^{-1/2} \nabla \ell(\theta_t)
\end{align*} 

First we evaluate each of the four optimization procedures described above. We
use a grid search to find the optimal hyperparameter settings for each
optimizer. The training curves for the optimizer with hyperparameters set to
achieve fastest convergence are plotted in Figure \ref{fig:allcurves} in the
appendix. Notice that the NAG and AdaGrad updaters far outstrip the others,
although are neck and neck with each other.

The network objective likely suffers from long narrow ravines leading towards
local optima surrounded with walls of high curvature. This hypothesis motivates
the use of momentum to encourage travel along the basin. AdaGrad also works to
this affect by penalizing directions which we travel in large magnitudes often
and encouraging directions with small but consistent gradients. AdaGrad
converges faster than momentum methods in the early stages of learning;
however, eventually suffers from the problem of diminishing common directions
of travel even if these directions are along the basin of a ravine. To achieve
convergence along a ravine, we need a form of acceleration along directions of
common travel which AdaGrad will not provide. 

The above discussion motivates combining the best of both the AdaGrad and
NAG optimizers to form an accelerated {\it and} adaptive gradient optimizer. We
initially attempt to do this with a simple reformulation given by
\begin{align*}
&v_{t+1} = \mu v_t - \eta \nabla \ell (\theta_t + \gamma v_t) \\
&G_{t+1} = G_{t} + \diag(\nabla \ell (\theta_t))^2 \\
&\theta_{t+1} = \theta_t + G_{t+1}^{-1/2} v_{t+1}
\end{align*}
where we use the descent direction $v$ given by NAG and rescale this direction
using the AdaGrad preconditioning matrix $G$. Results can be seen in Figure
\ref{fig:bestcurves} in the appendix. The accelerated AdaGrad seems to converge
slightly faster on the MNIST training data, in particular during the later
stages of training.  However, since both algorithms converge rapidly on the
small MNIST dataset, before making any claims we must rigorously test this
method on harder problems.

