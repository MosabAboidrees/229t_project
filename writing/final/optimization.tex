\section{Optimization}

We consider two families of stochastic optimization algorithms. Traditionally
used in optimizing deep neural networks, the first class is that of momentum
based methods which have recently been generalized and included the broader
group of accelerated gradient algorithms \cite{sutskever_2013}. The second class
includes variations on the adaptive gradient algorithms as presented in
\cite{duchi_2011}.

\subsection{Accelerated Gradient}

The accelerated gradient methods are a generalization of the basic stochastic
gradient descent (SGD) optimization procedure. Let $\ell(\theta)$ be the loss
function of the network with parameters $\theta$. We use accelerated SGD
methods that update the parameters at time $t$ according to
\begin{equation}
\begin{aligned}
&v_{t+1} = \mu v_t - \eta \nabla \ell (\theta_t + \gamma v_t) \\
&\theta_{t+1} = \theta_t + v_{t+1}
\end{aligned}
\end{equation}
where $\eta$ is the learning rate, and the parameter $\mu$, known as the
momentum parameter, dictates how much gradient history we take into account at
every update. If we set $\mu = 0$ we recover plain SGD, and setting $\mu = 1$
uses the full gradient history at every update.  The $\gamma$ parameter is
either active and set as $\gamma = \mu$ or inactive and set as $\gamma = 0$.
When $\gamma$ is inactive, we have typical SGD with ``momentum'', also known as
classical momentum (CM). On the other hand, when $\gamma$ is active the
procedure is known as Nesterov's accelerated graient (NAG).

A common belief is that network objectives likely suffer from long narrow
ravines leading towards local optima surrounded with walls of high curvature.
This hypothesis motivates the use of momentum to encourage persistent
directions of travel along the basin and suppress unwanted oscillation.

\subsection{Adaptive Gradient}

We study modifications to the adaptive gradient algorithm (AdaGrad) with the
diagonal preconditioning matrix $G$. The update can be written as
\begin{equation}
\begin{aligned}
&G_{t+1} = G_{t} + \diag(\nabla \ell (\theta_t))^2 \\
&\theta_{t+1} = \theta_t - G_{t+1}^{-1/2} \nabla \ell(\theta_t)
\end{aligned} 
\end{equation}

In some sense AdaGrad achieves a similar affect as NAG by penalizing
oscillating directions in which we take large steps and encouraging directions
with small but consistent gradients. However, in other aspects the optimization
routines behave quite differently. A simple property of AdaGrad which as we
show later can drasitcally affect optimization is the nondecreasing
monotinicity of the matrix $G$. This leads AdaGrad to penalize large yet
consistent directions of travel in the optimization process. This proves
locally favorable in most of the models studied in the next section, but
globally the behaviour results in better performance of NAG routine over the
vanilla AdaGrad algorithm.

This motivates two simple variations of the AdaGrad algorithm. The first is to
replace the square root on the matrix $G_t$ with a function which grows less
quickly (e.g. cube root). The second is to decay $G_t$ by a factor of $\gamma
\in (0,1]$ before including it in the update for $G_{t+1}$. This is close to
the AdaDelta algorithm presented in \cite{zeiler_2012}, but there they take a
convex combination of the two terms in the update of $G_{t+1}$ whereas here we
only decay the gradient history term.

% Discuss convergence properties of each method?
