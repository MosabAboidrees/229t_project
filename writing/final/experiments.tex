\section{Experiments}

\subsection{Setup}
In most experiments performed, we use a standard multilayer neural network with
dense connections between each layer of hidden activations. We study two
classes of networks. The first class is the standard densely connected
classification network trained with the cross entropy loss. For a single
training input and label $(x^{(i)},y^{(i)})$, the loss is given by
\begin{align}
\ell (x^{(i)},y^{(i)}) = - \sum_{k=1}^K \1\{y^{(i)} = k\} \log p_{i,k}
\end{align}
where $K$ is the number of classes and $p_{i,k}$ is the probability the model
assigns to example $i$ taking on label $k$. We also study the autoencoder
network which attempts to reconstruct the input after nonlinearly projecting it
to a much lower dimension. We use the squared loss to train this network, which
on input $x^{(i)}$ is given by
\begin{align}
\ell(x^{(i)}) = \frac{1}{2} \|x^{(i)} - h(x^{(i)})\|_2^2
\end{align}
and $h$ is the function which maps the network input to output.

The number of hidden layers and size of the networks vary; for the MNIST and
CIFAR classification experiments described below we use between 2 - 4 hidden
layers with 200 units each. The autoencoder is only trained on MNIST, and we
use the same 1.6 million parameter, 7 hidden layer network as in
\cite{hinton_2006} which is known to be a difficult to optimize benchmark. The
leaky rectified linear nonlinearity ($f(x) = \max\{0.01x,x\}$) is applied at
each hidden unit.  Most computation is accelerated using GPUs and the Gnumpy
library for Python \cite{tielemen_2010}.

We experiment mostly with the MNIST handwritten digits dataset which consists
of 60,000 training images and 10,000 test images. The images are all
$28\times28$ pixels; however, for computational efficiency, prior to training
classification networks we project the data onto the leading principal
components (between 50 and 100) and use this as the input. We also ran
experiments on the CIFAR-10 object recognition dataset \cite{krizhevsky_2009}.
This benchmark consists of 50,000 training images and 10,000 test images all
$32 \times 32$ and RGB. We first grayscale the images then use the same
preprocessing as that of the MNIST data.

\subsection{Results}

As a preliminary, we grid searched over all learning rate and momentum
hyperparameters to select the best for each optimization strategy including
vanilla SGD and CM. The standard SGD and CM (e.g.  $\gamma=0$ in \ref{nag})
optimization algorithms did not converge nearly as quickly as variations of
AdaGrad and the NAG optimizers, thus we restrict our attention to those two in
the following results and analysis.


\begin{figure}[ht]
\begin{center}
\centerline{\includegraphics[width=\columnwidth]{mnist_nn}}
\caption{Learning curves for the classification network on MNIST. The network
has 4 hidden layers each with 200 hidden units. The y-axis displays a windowed
average of the crossentropy cost after each parameter update. Ada3 is the cube
root version of AdaGrad as discussed in section \ref{adagrad}.}
\label{mnist_nn}
\vskip -.4in
\end{center}
\end{figure}


\begin{figure}[ht]
\begin{center}
\centerline{\includegraphics[width=\columnwidth]{cifar_nn}}
\caption{Learning curves for the classification network on CIFAR-10. Details
are the same as Figure \ref{mnist_nn}.}
\label{cifar_nn}
\vskip -.4in
\end{center}
\end{figure}

In both classification networks trained on MNIST and CIFAR-10 (cf. Figures
\ref{mnist_nn} and \ref{cifar_nn} respectively), NAG and standard AdaGrad
perform comparably well. However, by choosing $\gamma$ wisely for the decayed
version of AdaGrad, we can achieve much faster convergence particularly after
the initial stages of fast learning. The oscillations in Figure \ref{cifar_nn}
are somewhat inexplicable. For the NAG updater these oscillations could be due
to the same observations as \cite{odonoghue_2012}, suggesting that the momentum
is on the high side. However, we observe fastest convergence with the momentum
set approximately in the interval $\mu \in [0.9,0.99]$. This also suggests that
an adaptive restart of the velocity $v_t$, (i.e. resetting the velocity to zero
on some indicator such as an increase in the objective function) could achieve
faster convergence \cite{odonoghue_2012}. 

\begin{figure}[ht]
\begin{center}
\centerline{\includegraphics[width=\columnwidth]{mnist_ae}}
\caption{Learning curves for the autoencoder used in \cite{hinton_2006}. The
network has 7 hidden layers of sizes 1000, 500, 250, 30, 250, 500, and 1000
with 1.6 million trainable parameters. The y-axis is a windowed average of
mean-squared reconstruction error, the objective used to train the network.}
\label{mnist_ae}
\vskip -.4in
\end{center}
\end{figure}

In order to robustly test the decayed version of the AdaGrad algorithm, we
trained an autoencoder with 7 hidden layers. Again, with the correct setting of
$\gamma$ in the decayed version of AdaGrad, we notice much faster convergence,
although the difference between NAG is not as pronounced in the later stages of
learning. The number of iterations needed to train this network causes the
vanishing learning rate problem for AdaGrad to be more severe even for the
cubic version. 

Figures xxx  in the appendix demonstrate another nice property of the decayed
version of AdaGrad, namely robustness to learning rate. Notice that in Figure
xx if we do not tune the learning rate well, AdaGrad will simply not converge
and NAG (cf. Figure xx), while slightly more robust, will take a long time.
However, as demonstrated in Figure xx, the decayed AdaGrad enjoys a robustness
to a wide range of learning rates, converging quickly to nearly the same point
at the same rate after the initial stages of learning. 
