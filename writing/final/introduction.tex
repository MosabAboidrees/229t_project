\section{Introduction}

Multilayer neural networks and other deep models have gained considerable
traction over the past several years demonstrating state-of-the-art performance
in many applications. Correspondingly, much work has been done to improve
methods for optimizing these functions. This has proven difficult due to the
highly nonconvex landscape of the typical multilayer objective and the supposed 
ill-conditioned objectives. Given growth in dataset sizes and
that these highly expressive models tend to be used in the large data setting,
stochastic optimization methods are common. However, stochastic methods tend to
come with a host of hyperparameters and much tuning is involved to get these to
work well which has lead to general disagreement as to the supieriority of a
particular algorithm. 

Until recently stochastic gradient descent (SGD) with momentum has been the
standard stochastic optimizer used with deep neural networks (DNNs)
\cite{hinton_2010}; however, other stochastic methods are beginning to be
adopted including AdaGrad \cite{duchi_2011} and variations as well as the
formulation of Nesterov's accelerated gradient (NAG) as presented in
\cite{sutskever_2013}. To the best of our knowledge, we do not know of any work
which gives a thorough comparison of these algorithms and demonstrates when one
or the other should be applied. There has been some work in automating the
hyperparamter tuning process for a given algorithm \cite{snoek_2012}, yet this
does not answer the question of which algorithm should be used in the first
place and furthermore still requires the training of many models which can be
prohibitive. 

Furthermore, recent work has begun to attempt an analytical understanding of
these types of heirarchichal models; however, often the results are derived using
simplifications or modifications of a typical DNN which make them not
applicable in practice \cite{saxe_2013}. 

We attempt to prescribe a stochastic optimization procedure for the typical
multilayer neural network objective. We develop a deeper
understanding of the most common objectives and architectures used in this
setting. We apply this analysis to motivate the selection of optimization
procedures we study and attempt to give a thorough comparison of a few
competing algorithms using existing theory and empirical results. 

